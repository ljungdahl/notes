
\subsection{Basics}
\emph{Main refs: Shankar, Principles of QM \& Beiser, Concepts of Modern Physics}.\\
Recall the principal quantum number
\eqa{principquant1}
    {
      n = k + l + 1.
    }
    The $k$-index comes from the series solution to the radial part of the hydrogen Schrödinger equation, and the $l$-index from the actual form of the equation. The indices can take values $k=0,1,2,\dots$ and $l=0,1,2,\dots$, and so the principal quantum number can take values $n=1,2,3,\dots$, and so on. This also imposes restrictions on $l$ for every $n$, since
    \eqa{lquantnr}
    {
      l=k-n-1.
    }
    Thus $l = n-1,n-2,\dots,1,0$. This means that quantum states of a certain $n$ are \emph{degenerate}, there are several states for a given $n$, namely those with different $l$. Shankar says that this indicates the Hamiltonian having more symmetries than just rotational invariance [I NEED TO UNDERSTAND THIS!]. The degeneracy, ie the number of states for each $n$, is calculated by
    \eqa{degenform}
    {
      \sum_{l=0}^{n-1} (2l+1) = n^2.
    }
    Proof by induction probably. The $l$ quantum number is called the orbital quantum number and is related to the angular momentum $L$ of the electron. More precisely,
    \eqa{angularmom1}{
      L = \sqrt{l(l+1)}\hbar,
    }
    which we can deduce from identifying orbital kinetic energy $KE_{orbital} = \frac{1}{2}m\v{v}_{orbital}^2$ with $\hbar^2l(l+1)/(2mr^2)$, and noting that $L = m\v{v}_{orbital}r$ [cf radial part of SE].\\
\indent The principal quantum number is related to the energy eigenstates:
\eqa{energyeigens}
    {
      E_n = \dfrac{-me^4}{2\hbar^2n^2}, \hspace{5pt} n=1,2,3,\dots
      }
    In atomic physics we denote the angular momentum states for $l=0,1,2,3,4,5,6,\dots$ as $s,p,d,f,g,h,i,\dots$, and so we can talk about the state $2s$ for example. By this we mean the state with $n=2$ and $l=0$. In the same way the state $3p$ is the state with $n=3$ and $l=1$ and so on.\\
    \indent But what about the quantum number $m$? This is the ``magnetic quantum number'', it determines the \emph{component} of the electron angular momentum $\v{L}$ in the \emph{direction} of an external magnetic field. This is referred to as ``space quantisation'', and $m$ can take values $m=0,\pm 1,\pm 2,\dots,\pm l$, for any given $l$. The angular momentum quantum number $l$ is only related to the \emph{magnitude} $|\v{L}| = L$ of the electron angular momentum. There are a lot of very ``quantum'' subtleties here that I should go through again (main ref Griffiths probably).\\
    \indent Anyhow, the three quantum numbers $n,l,m$ specify the electron wave-function $\Psi_{nlm}$. These wave-functions are called the ``orbitals'', and are the precise mathematical description of the possible quantum states for a bound electron in an atom, even though we haven't explicitly written it down yet. More correctly $|\Psi_{nlm}(r,\theta,\phi)|^2$ is the spatial probability of an electron in the $nlm$-state. 
