\subsection{Commutation relations}
\newcommand{\vl}{\v{l}}
\newcommand{\rr}{\v{r}}
\newcommand{\pp}{\v{p}}
We consider a point particle (say an electron) whose motion can be described by some coordinate system. For simplicity let's say cartesian coordinates, so we can describe the particle as having position $\v{r} = (x,y,z)$ and linear momentum $\v{p} = (p_x,p_y,p_z)$ at any given time.
We define orbital angular momentum in quantum mechanics in the same way we do in classical physics. The orbital angular momentum $\v{l}$ about the origin of the chosen coordinate system is given by the cross-product%\footnote{\import{./}{angmom1_footnote1.tex}}
\eqa{orbital_angmom_def_crossprod}
{
    \vl = \rr \times \pp.
}
In quantum mechanics we consider $\rr$ and $\pp$ as operators (and each of their components is an operator), and the linear momentum operator can be replaced by
\eqa{linmomentum_op_replacement}
{
    \pp \rightarrow -\i\nabla.
}
So we write the angular momentum operator by components as
\eqa{angmom_operator_components}
{
    l_x &= yp_z-zp_y = &-\i\qty(y\dv{z}-z\dv{y}),\\
    l_y &= zp_x-xp_z = &-\i\qty(z\dv{x}-x\dv{z}),\\
    l_z &= xp_y-yp_x = &-\i\qty(x\dv{y}-y\dv{x}).
}
In quantum mechanics we know that the components of $\rr$ and $\pp$ does not commute in general, in fact we have a defining commutation relation of quantum mechanics:
\eqa{xp_commutator_1}
{
    \commutator{x_i}{p_j} = \i\hbar\delta_{ij},
}
and $\commutator{x_i}{x_j} = \commutator{p_i}{p_j} = 0$. In words: The operators $x_i$ and $p_j$ does \emph{not} commute "along the same axis" (but they do along different). One can suspect that this has consequences for the commutation relations of the angular momentum operators $l_x,l_y,l_z$. One such relation would for example be 
\eqa{angmom_commut_rel_example1}
{
    \commutator{l_x}{l_y} = l_x l_y-l_y l_x = \qty(yp_z-zp_y)\qty(zp_x-xp_z) - \qty(zp_x-xp_z)\qty(yp_z-zp_y).
}
