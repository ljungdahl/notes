\newcommand{\vl}{\v{l}}
\newcommand{\rr}{\v{r}}
\newcommand{\pp}{\v{p}}
\newcommand{\vs}{\v{s}}
\newcommand{\vj}{\v{j}}
We consider a point particle (say an electron) whose motion can be described by some coordinate system. For simplicity let's say cartesian coordinates, so we can describe the particle as having position $\v{r} = (x,y,z)$ and linear momentum $\v{p} = (p_x,p_y,p_z)$ at any given time.
We define orbital angular momentum in quantum mechanics in the same way we do in classical physics. The orbital angular momentum $\v{l}$ about the origin of the chosen coordinate system is given by the cross-product%\footnote{\import{./}{angmom1_footnote1.tex}}
\eqa{orbital_angmom_def_crossprod}
{
    \vl = \rr \times \pp.
}
In quantum mechanics we consider $\rr$ and $\pp$ as operators (and each of their components is an operator), and the linear momentum operator can be replaced by
\eqa{linmomentum_op_replacement}
{
    \pp \rightarrow -\i\hbar\nabla.
}
So we write the angular momentum operator by components as
\eqa{angmom_operator_components}
{
    l_x &= yp_z-zp_y = &-\i\hbar\qty(y\dv{z}-z\dv{y}),\\
    l_y &= zp_x-xp_z = &-\i\hbar\qty(z\dv{x}-x\dv{z}),\\
    l_z &= xp_y-yp_x = &-\i\hbar\qty(x\dv{y}-y\dv{x}).
}
In quantum mechanics we know that the components of $\rr$ and $\pp$ does not commute in general, in fact we have a defining commutation relation of quantum mechanics:
\eqa{xp_commutator_1}
{
    \commutator{x_i}{p_j} = \i\hbar\delta_{ij},
}
and $\commutator{x_i}{x_j} = \commutator{p_i}{p_j} = 0$. In words: The operators $x_i$ and $p_j$ does \emph{not} commute "along the same axis" (but they do along different). One can suspect that this has consequences for the commutation relations of the angular momentum operators $l_x,l_y,l_z$. One such relation would for example be 
\eqa{angmom_commut_rel_example1}
{
    \commutator{l_x}{l_y} = l_x l_y-l_y l_x = \qty(yp_z-zp_y)\qty(zp_x-xp_z) - \qty(zp_x-xp_z)\qty(yp_z-zp_y),
}
and we could evaluate this by expanding the parentheses and employing the $x,p$ commutation relations and the form of $p$ as a differential operator. But we could also be a bit more systematic and also a bit more illustrative by figuring out commutators between the $l_i$, $x_i$ and $p_i$. Then we use the algebraic properties of the commutator to figure out the commutators between angular momentum operators (following Dirac's treatment).\\
\indent The algebraic rules follows from the definition (these rules also apply to classical Poisson brackets)
\eqa{commutator_def}
{
    \commutator{u}{v} = uv-vu,
}
and are 
\eqa{commutator_algebraic_rules1}
{
    \commutator{u_1+u_2}{v} &= \commutator{u_1}{v}+\commutator{u_2}{v},\\
    \commutator{u}{v_1+v_2} &= \commutator{u}{v_1}+\commutator{u}{v_2},\\
    \commutator{u_1 u_2}{v} &= \commutator{u_1}{v}u_2+u_1\commutator{u_2}{v},\\
    \commutator{u}{v_1 v_2} &= \commutator{u}{v_1}v_2+v_1\commutator{u}{v_2}.
}
To proceed we need to perform a bit of algebra, and it's not very useful to do this in a text document. I will append handwritten work to the end of this document. 
Let us just simply state the results here. We have the commutation relations
\eqa{angmom_commut_relations1}
{
    \commutator{l_x}{l_y} &= \i\hbar l_z,\\
    \commutator{l_z}{l_x} &= \i\hbar l_y,\\
    \commutator{l_y}{l_z} &= \i\hbar l_x.
}
We can write this compactly by using the Levi-Civita symbol defined by
\eqa{levicivita_definition}
{
    \levciv = \left\{\mqty{
        &1& \text{ if } (i,j,k) \text{ is an \emph{even} permutation of } (1,2,3), \\
        &-1& \text{ if } (i,j,k) \text{ is an \emph{odd} permutation of } (1,2,3), \\
        &0& \text{ if any two or more of the indices are the same.}}\right.
}
Then the commutation relations simply becomes
\eqa{leviciv_angmom_commut_rels}
{
    \commutator{l_i}{l_j} = \i\hbar\levciv l_k.
}
We have the same relation for orbital angular momentum in classical mechanics, using Poisson brackets: $\qty{l_i,l_j} = \levciv l_k$.\\
\indent But in quantum mechanics we also have spin angular momentum $\vs$, and spin does not have any classic analogue. We cannot describe spin as a cross product between position and momentum vectors. It is useful to introduce a general angular momentum
\eqa{general_angular_momentum}
{
    \vj = j_x \v{\hat{x}}+j_y \vu{y}+j_z \vu{z},
}
and say that quantum angular momentum operators are \emph{defined} by the commutation relations
\eqa{general_commutation_relations1}
{
    \commutator{j_i}{j_j} = \i\hbar \levciv j_k.
}
All the properties we need to describe the physics of quantum angular momentum follows from this definition.