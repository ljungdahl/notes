\subsection{The Hartree-Fock equations}
\newcommand{\hHF}{h_{\text{HF}}}
\newcommand{\uHF}{u_{\text{HF}}}
\newcommand{\occs}{\text{occ}}
The variational principle tells us that we can find the ``best'' approximation to the ground state of some system, with Hamiltonian $H$, by minimising the expectation value of the
energy,
\eqa{varprinc1}{
  \braket{E} = \braket{0|H|0}.
  }
If we look to find the ground state Slater determinant $\ket{\alpha}$ of an atomic system we look at the energy expectation value (ignoring magnetic effects)
\eqa{nrgexp1}{
  \braket{E} = \braket{\alpha|\sum_{i=1}^N\left(-\dfrac{1}{2}\nabla_i^2 -\dfrac{Z}{r_i}\right)+\sum_{i<j}^N \dfrac{1}{r_{ij}}|\alpha},
  }
with the condition that this expectation value is \emph{stationary} with respect to small changes in the orbitals of $\ket{\alpha} = \ket{\slater{abcd\dots n}}$. We will use this condition to derive the Hartree-Fock (HF) equations.\\
\indent The small change in the orbital $\ket{a}$ can be expressed by
\eqa{change1}{
  \ket{a} \rightarrow \ket{a}+\eta\ket{r},
}
where $\eta$ is a small real number. Earlier we said that a Slater determinant in which orbital $a$ is replaced by excitation $r$ is labeled $\ket{\alpha_a^r}$, and so this small change in the orbital can be written a small change in the Slater determinant,
\eqa{slaterchange1}{
  \ket{\alpha} \rightarrow \ket{\alpha}+\eta\ket{\alpha_a^r}.
  }
Thus the corresponding energy expectation value after the small change is
\eqa{energychange1}{
  \left(\bra{\alpha}+\eta\bra{\alpha_a^r}\right)H\left(\ket{\alpha}+\eta\ket{\alpha_a^r}\right) = \braket{E} + \eta\left(\braket{\alpha_a^r|H|\alpha}+\braket{\alpha|H|\alpha_a^r}\right),
}
where we the terms quadratic in $\eta$, since $\eta$ is taken to be small ``enough''. We now note that $H$ is Hermitian so that the two ``change''-terms are equal, thus the RHS in equation \eqref{energychange1} becomes
\eqa{energychange2}{
  \braket{E} + \eta\left(\braket{\alpha_a^r|H|\alpha}+\braket{\alpha|H|\alpha_a^r}\right) = \braket{E} + 2\eta\braket{\alpha_a^r|H|\alpha}.
}
So the condition that the energy expectation value is stationary (ie doesn't change) due to a small change in orbitals, amounts to saying that
\eqa{brillouincond}{
  \braket{\alpha_a^r|H|\alpha} = 0.
  }
This is called \emph{Brillouin's theorem} (1933,1934), and means that there are no matrix elements of $H$ between $\ket{\alpha}$ and states with \emph{single substitution}. We will see that it implies
that there is no first-order mixing of such states (whatever that might mean).\\
\indent We can write this condition in a more direct way, using the expressions for single- and two-particle matrix elements of $N$-electron systems from the previous section. Writing out the atomic Hamiltonian explicitly the condition becomes
\eqa{hamiltbrillcond}{
  \braket{\alpha_a^r|\sum_{i=1}^N\left(-\dfrac{1}{2}\nabla_i^2 -\dfrac{Z}{r_i}\right)|\alpha}+\braket{\alpha_a^r|\sum_{i<j}^N \dfrac{1}{r_{ij}}|\alpha} = 0,
  }
where the first term is made up of sums of one-particle operators, and the second term is a sum of two-particle operators. Using the expressions for matrix elements of such operators in Equation \eqref{FGDiff2_1} we get that the condition can be written
\eqa{operatorbrillcond1}{
  \braket{r|-\dfrac{1}{2}\nabla_i^2 -\dfrac{Z}{r_i}|a}+
  \left[\sum_{b}^{\text{occ}}\left(\braket{rb|\dfrac{1}{r_{ij}}|ab}-\braket{br|\dfrac{1}{r_{ij}}|ab}\right)\right] = 0.
}
We should note that the label $r$ for the excited orbital is not related to $r_i$ or $r_{ij}$. LM drops the index on $r_i$ and sets $r_{ij}$ to $r_{12}$ from this point onwards, but it's not clear to me yet why this can be done? Maybe we are putting the $i$-dependence into $\hHF$,$\uHF$ (see below)? No, it rather is that electron $i$ is the electron of orbital $\ket{a}$, and summing over occupied states $b$ we are summing over electrons $j$?\\
\indent We further want to simplify things by introducing a Hartree-Fock operator ($\hHF$) and Hartree-Fock potential ($\uHF$) by defining
\eqa{hfpotsoperators1}{
  \hHF &= -\dfrac{1}{2}\nabla^2 -\dfrac{Z}{r}+\uHF,\\
  \braket{i|\uHF|j} &= \sum_b^{\occs}\left(\braket{ib|\dfrac{1}{r_{12}}|jb}-\braket{bi|\dfrac{1}{r_{12}}|jb}\right).
  }
Then we can write the Brillouin condition as
\eqa{hfbrill}{
  \braket{r|\hHF|a} = 0.
  }
We can show (I want to do this later) that $\hHF$ is Hermitian and unitary, and thus we can find a basis (orbitals) $\{\ket{a'}\}$ such that $\hHF$ is diagonal, and
\eqa{hfeqs1}{
  \hHF\ket{a'} = \varepsilon_a'\ket{a'}.
  }
This eigenvalue equation is called in LM ``the normal form of the Hartree-Fock equation''. Dropping the primes for convenience (we remember that we are dealing with the basis in which $\hHF$ is diagonal) we can write explicitly
\eqa{hfeqsexpl1}{
  \left(-\dfrac{1}{2}\nabla^2-\dfrac{Z}{r}+\uHF\right)\ket{a} = \varepsilon_a \ket{a}.
  }
This equation is nice since each term can be given a simple physical interpretation. The first term is the kinetic energy of the electron, and the second term represents the attraction due to the atomic nucleus. The third term, the potential $\uHF$, represents the average effect of the Coulomb interaction of the electron represented by $\ket{a}$ with the rest of the atomic electrons, \emph{including the exchange interaction}.\\
\indent Next up is to identify the physical meaning of the eigenvalue $\varepsilon_a$ in the normal form Hartree-Fock Equation \eqref{hfeqsexpl1}.

\subsection{Koopmans' Theorem}
The total energy of an atomic state represented by a Slater determinant $\ket{\alpha}$ is represented by the expectation value
\eqa{totalexp}{
  \braket{E_{\text{atom}}} = \braket{\alpha|H|\alpha}.}
We know the explicit form of the Hamiltonian, a sum of one- and two-particle operators, so we can play the same game as we did when simplifying $\braket{\alpha_a^r|H|\alpha}$, but with diagonal elements. From equations \eqref{Fdiagelemnot} and \eqref{Gdiagelems2} we get that
\eqa{energyexpvalatom1}{
  \braket{E_{\text{atom}}} = \braket{\alpha|\sum_{i=1}^N\left(-\dfrac{1}{2}\nabla_i^2 -\dfrac{Z}{r_i}\right)+\sum_{i<j}^N \dfrac{1}{r_{ij}}|\alpha}
  }
becomes
\eqa{energyexpvalatom2}{
  \braket{E_{\text{atom}}} = \sum_{b}^{\occs}\braket{b|\left(-\dfrac{1}{2}\nabla^2-\dfrac{Z}{r}\right)|b}\\
  +\dfrac{1}{2}\sum_{bc}^{\occs}\left(\braket{bc|r_{12}^{-1}|bc}-\braket{cb|r_{12}^{-1}|bc}\right).
  }
Consider now the same energy expectation value $\braket{E_{\text{ion}}}$ for the same atom but with orbital $\ket{a}$ removed; an atomic ion. We assume that this removal doesn't affect the other orbitals in any way. The difference between the expectation values then is exactly the terms regarding orbital $\ket{a}$ in Equation \eqref{energyexpvalatom2},
\eqa{expvaldiff1}{
  \braket{E_{\text{atom}}}-\braket{E_{\text{ion}}} = \braket{a|\left(-\dfrac{1}{2}\nabla^2-\dfrac{Z}{r}\right)|a}+\sum_b^\occs \left(\braket{ab|r_{12}^{-1}|ab}-\braket{ba|r_{12}^{-1}|ab}\right).
  }
The factor $\frac{1}{2}$ disappears from the second term since we have a contribution from $a$ in both the sum over $b$ and over $c$ in Equation \eqref{energyexpvalatom2}, and $\braket{ab|g|ab} = \braket{ba|g|ba}$ and $\braket{ba|g|ab} = \braket{ab|g|ba}$. Is this correct?\\
\indent But now we need to recall how we defined the HF operator $\hHF$, Equation \eqref{hfpotsoperators1}! We now see that the RHS of Equation \eqref{expvaldiff} is in fact $\braket{a|\hHF|a}$. And $\hHF\ket{a} = \varepsilon_a\ket{a}$ (recall that we can express orbital $a$ in a basis such that $\hHF$ is diagonal). So the difference in energy between atom and ion is
\eqa{expvaldiff2}{
  \braket{E_{\text{atom}}}-\braket{E_{\text{ion}}} = \braket{a|\hHF|a} = \braket{a|\varepsilon_a|a} = \varepsilon_a\braket{a|a} = \varepsilon_a.
}
What we have found here is that the eigenvalue of the Hartree-Fock (HF) operator is the negative work required to remove one electron from the atomic system. Another way to phrase this is to say that the eigenvalue of $\hHF$ is the negative of the \emph{binding energy}. This result is called Koopman's Theorem (Koopmans 1933).\\
\indent The important assumption we used here with the removal of one electron not affecting the other orbitals is not physically accurate. The system should ``readjust'' or ``relax'' to the removal of an electron. The theorem is only valid if we can neglect the effect of this ``relaxation''. 
