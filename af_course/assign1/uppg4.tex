\subsection{a)}
För att vi betraktar triplet-övergångar på typen $^3L_J \rightarrow  ^3L'_{J'}$.

\subsection{b)}
Antar vi att det bara är sluttillståndet som splittras har kan vi säga att varje linje i (a)-figuren
svarar mot en energi $h\nu_a = E_{J_a}-E_{\text{init}}$. Skillnaden mellan två av de här fotonenergierna är då
\eqa{fotondiff1}{
\epsilon_{ab} = h\nu_a-h\nu_b = E_{J_a}-E_{\text{init}}-E_{J_b}+E_{\text{init}} = E_{J_a}-E_{J_b}.
}
Vi vill försöka bestämma alla tre kvanttalen $J$, det ger oss direkt $L$ eftersom vi redan vet $S$ (triplet ger $S=1$).
I fallet då $J_a = J_b-1$ kan vi använda Landés intervallregel $E_{J_a}-E_{J_b} = \beta J_a$. Om vi beräknar $\epsilon_{ij}$
för skillnaden mellan alla linjer kan vi ta kvoten mellan skillnaderna. I fallet då Landés intervallregel är uppfylld får vi
något på formen
\eqa{fotondiffq1}{
    \dfrac{E_{J_a}-E_{J_b}}{E_{J_b}-E_{J_c}} = \dfrac{J_a}{J_b} = \dfrac{J_a}{J_a-1}.
}
Vi testar helt enkelt det här för $\lambda_a = 4585.90$ Å, $\lambda_b = 4581.41$ Å, $\lambda_c = 4578.57$ Å, t. ex med ett
python script. Vi får skillnaderna
\eqa{epsilonz1}{
    \epsilon_{ab} &= -0.00264965 \text{ eV},\\
    \epsilon_{bc} &= -0.00167863 \text{ eV},\\
    \epsilon_{ac} &= -0.00432829 \text{ eV}.
    }
Kvoterna blir (jag använder $\epsilon_{ab} = -\epsilon_{ba}$)
\eqa{quotients1}{
    \epsilon_{ab}/\epsilon_{bc} &\approx 1.58,\\
    \epsilon_{bc}/\epsilon_{ca} &\approx -0.39,\\
    \epsilon_{ba}/\epsilon_{ac} &\approx -0.61.
}
Vi ser att $\epsilon_{ab}/\epsilon_{bc}$ ungefär är $3/2$, medan de andra kvoterna är negativa. För tre $J$-värden har vi bara en möjlig
kvot enligt ekvation \eqref{fotondiffq1}, och den måste per definition vara positiv. Så vi kan nu dra slutsatsen att $J_a = 3$, $J_b = 2$ och då $J_c = 1$.
Eftersom $J = L-S,\dots,L+S$ får vi då (med känt $S=1$) direkt $L=2$. Vi betraktar alltså övergångar från ett initialt tillstånd
(som vi inte kan urskilja splittring i) till en triplet $^3D_1,^3D_2,^3D_3$.

\subsection{c)}
Anledningen till att vi bara ser sex övergångar är urvalsreglerna. Om man sätter sig ner och klurar med dem upptäcker vi att 
om vi skulle ha en övergång mellan termer med samma $L$ skulle vi få åtta möjligheter. Alltså har vi övergångar mellan olika $L$.
Urvalsreglerna säger oss också att de tre linjer i spektrat som ligger nära varandra alla är övergångar som har samma $J$-kvanttal på initial \emph{eller}
sluttillståndet. Vi kan alltså använda oss av samma idé som i uppgift b) på dessa tre linjer för att börja någonstans. Nu får vi 
$\lambda_a = 4456.61$ Å, $\lambda_b = 4455.88$ Å, $\lambda_c = 4454.77$ Å. I det här fallet visar det sig att kvoten som är en vettig Landé intervall-kvot 
är 
\eqa{validquotient1}{
    \dfrac{E_{J_c}-E_{J_b}}{E_{J_b}-E_{J_a}} = \dfrac{J_c}{J_b} = {J_c}{J_c-1} \approx 1.5 \Longrightarrow J_c = 3,
}
alltså har vi igen triplett $D$ termer. Frågan nu är om det är i den övre eller undre termen i övergången som vi beräknat kvanttalen för. Det här kan vi lista ut genom att betrakta fotonenergierna (våglängderna i spektrumet). $\lambda_c$ är den kortaste våglängden av de tre, och den svarade mot det högsta $J$-kvanttalet i vår kvot.
Alltså ökar energin med $J$, vilket måste betyda att vi beräknat $J$ för den \emph{övre} termen, som alltså är en triplett $D$. Nu behöver vi lista ut om vi 
går till en $P$ eller $F$ term.