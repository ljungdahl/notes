\newcommand{\vmu}{\hat{\bm\mu}}
\newcommand{\vB}{\v{B}}
\newcommand{\JLS}{J M_J L S}

Uttrycket för första ordningens energiskift som vi är ute efter är 
\eqa{zeeman1}{
    \Delta E = g_J \mu_B B_z M_J,
}
där $g_J$ är Landé $g$-faktorn
\eqa{landegfaktor1}{
    g_J = 1+\dfrac{J(J+1)-L(L+1)+S(S+1)}{2J(J+1)},
}
och $\mu_B$ är Bohr magnetonen, $B_z$ magnetfältets komponent i $z$-riktningen (vi antar $\vB = B_z\hat{\mathbf{e}}_z$).
 Vi antar ett svagt magnetfält så är termen proportionell mot $|\vB|^2$ i Hamiltonianen inte bidrar
nämnvärt. Vår störning är alltså Zeeman-Hamiltonianen $H' = -\vmu \cdot \vB$, och energin är i första ordningen
\eqa{zeeman2}{
    \expval{H'} = -\expval{\vmu \cdot \vB}.
}
Här är $\vmu = -\mu_B\qty(\vL + 2 \vS)$, där vi tagit $g_S = 2$. Vi antar att LS-koppling gäller och vi använder 
$\ket{\JLS}$ som bastillstånd för väntevärdet. Magnetfältsvektorn är ingen operator här, så vi betraktar först bara
väntevärdet
\eqa{expvalzeeman1}{
    -\expval{\vmu} = \mu_B \qty(\expval{\vL}+2\expval{\vS}).
}
Nu kan vi använda Wigner-Eckarts sats igen! Både $\vL$ och $\vS$ är i samma vektorrum som $\vJ = \vL + \vS$, så deras
väntevärden är proportionella mot $\vJ$ (kom ihåg att vi räknar väntevärden mellan $\bra{\JLS}$ och $\ket{\JLS}$):
\eqa{Lprop}{
    \expval{\vL} = \dfrac{\expval{\vL \cdot \vJ}}{J(J+1)}\expval{\vJ},
}
och motsvarande för $\expval{\vS}$. Då har vi alltså totalt
\eqa{expvalzeeman2}{
    \expval{H'} = \dfrac{\expval{\vL \cdot \vJ}+2\expval{\vS \cdot \vJ}}{J(J+1)}\mu_B\expval{\vJ}\cdot \vB.
}
Eftersom $\vJ \cdot \vB = J_z B_z$ och $\expval{J_z} = M_J$ får vi
\eqa{expvalzeeman3}{
    \expval{H'} = \qty[\dfrac{\expval{\vL \cdot \vJ}+2\expval{\vS \cdot \vJ}}{J(J+1)}]\mu_B B_z M_J.
}
Nu återstår det bara att visa att vi kan få Landés $g$-faktor \eqref{landegfaktor1} från faktorn i brackets. Det 
åstadkommer vi med lite algebra utifrån att $\vJ = \vL + \vS$ och $\vL \cdot \vS = \frac{1}{2}\qty(\vJ^2-\vL^2-\vS^2)$, 
dvs vi kan beräkna väntevärdena om vi uttrycker dem i termer av $\vJ^2,\vL^2,\vS^2$. Vi får att
\eqa{gfaktoralgeb1}{
    \expval{\vL \cdot \vJ}+2\expval{\vS \cdot \vJ} = \\
    \expval{\vL \cdot \qty(\vL + \vS)}+2\expval{\vS \cdot \qty(\vL+\vS)} = \\
    \expval{\vL^2+\vL\cdot\vS}+2\expval{\vS^2 +\qty(\vL \cdot \vS)} =\\
    \expval{\vL^2}+2\expval{\vS^2}+3\expval{\vL \cdot \vS} = \\
    \expval{\vL^2}+2\expval{\vS^2}+\dfrac{3}{2}\expval{\vJ^2-\vL^2-\vS^2} = \\
    \dfrac{3\expval{\vJ^2}-\expval{\vL^2}+\expval{\vS^2}}{2}.
}
Med $\expval{\vJ^2} = J(J+1)$, $\expval{\vL^2} = L(L+1)$ och $\expval{\vS^2} = S(S+1)$ får vi alltså att
\eqa{gfaktorhej1}{
    \dfrac{\expval{\vL \cdot \vJ}+2\expval{\vS \cdot \vJ}}{J(J+1)} = \dfrac{3J(J+1)-L(L+1)+S(S+1)}{2J(J+1)} = \\
    1+\dfrac{J(J+1)-L(L+1)+S(S+1)}{2J(J+1)} = g_J.
}
Då har vi visat att första ordningens energikorrektion från interaktion med magnetfält ges av
\eqa{finitofem}{
    \Delta E = \expval{H'} = g_J\mu_B B_z M_J.
}